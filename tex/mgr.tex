%% LyX 1.5.1 created this file.  For more info, see http://www.lyx.org/.
%% Do not edit unless you really know what you are doing.
\documentclass[11pt,oneside,polish,wide,pdftex]{mwbk}
\usepackage[T1]{fontenc}
\usepackage[latin2]{inputenc}
\usepackage{geometry}
\geometry{verbose,a4paper,tmargin=2cm,bmargin=2cm,lmargin=3.5cm,rmargin=2.1cm}
\pagestyle{uheadings}
\setcounter{secnumdepth}{3}
\setcounter{tocdepth}{3}
\usepackage{floatflt}
\usepackage{varioref}
\usepackage{amsmath}
\usepackage{graphicx}
\usepackage{setspace}
\onehalfspacing
\IfFileExists{url.sty}{\usepackage{url}}
                      {\newcommand{\url}{\texttt}}

\makeatletter

%%%%%%%%%%%%%%%%%%%%%%%%%%%%%% LyX specific LaTeX commands.
\newcommand{\noun}[1]{\textsc{#1}}

%%%%%%%%%%%%%%%%%%%%%%%%%%%%%% User specified LaTeX commands.
\usepackage{thumbpdf}

\usepackage{fix-cm}

\usepackage{type1ec}
\usepackage[T1]{fontenc}


%\usepackage[pdftex,man,opt]{wmstitle}
\usepackage[pdftex,type=mgr]{aghtitle}
\usepackage[bookmarks, bookmarksopen=true, bookmarksnumbered=true, colorlinks=true, pdftitle={System sterowania robotem mobilnym}, pdfauthor={Micha� Fita}]{hyperref}

\usepackage[protrusion=true,expansion=true, tracking=true, spacing=true, kerning=true, selected=true, babel=true]{microtype}

\usepackage{babel}
\makeatother

\begin{document}
\promotor{dr in�. Roman Krasowski}
\albumNumber{209034}


\title{System sterowania robotem mobilnym}


\author{Micha� Fita}

\maketitle
\tableofcontents{}


\chapter*{Wst�p}

S�owo {}``robot'' pojawia si� po raz pierwszy w sztuce, kt�rej autorem
jest czeski pisarz Karel �apek. Okre�lenie to szybko trafia pomi�dzy
ok�adki dzie� literatury fantastycznej, zmieniaj�c pierwotne znaczenie
ze sztucznie chodowanego cz�owieka na sztucznie konstruowanego cz�owieka,
by przez kolejne lata ewoluowa� do dziesiejszego znaczenia elektromechanicznej
maszyny wykonuj�cej r�norodne prace. Premiera czeskiej sztuki mia�a
miejsce w 1921 roku, a dzi� prawie wiek p�niej s�owo {}``robot''
jest tak powszechne, �e ma�o kto wie, jaka jest jego geneza. Tym bardzej
niewielu wie, �e termin {}``robotyki'' jako dziedziny bada� naukowych
stworzy� Isaac Asimov, kt�ry u�y� go w swoim opowiadaniu w 1942 roku.



\end{document}
